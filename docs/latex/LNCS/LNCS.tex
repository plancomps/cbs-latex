\documentclass[runningheads,fleqn]{llncs}
\usepackage{amsmath}
\usepackage{amssymb}

\usepackage{../../../_includes/cbs-latex-lncs}
\usepackage{url}

\usepackage{hyperref}
% If you use the hyperref package, please uncomment the following line
% to display URLs in blue roman font according to Springer's eBook style:
\renewcommand\UrlFont{\color{blue}\rmfamily}

\newcommand{\QUESTION}[1]
{\pagebreak[3]\begin{itemize}\item \emph{#1} \end{itemize}}

\begin{document}

\title{Fundamental~Constructs in Programming~Languages}

\subtitle{Extracts for testing MathJax}

\titlerunning{Fundamental Constructs}

\author{Peter~D.~Mosses\inst{1,2}\orcidID{0000-0002-5826-7520}}

\institute{Delft University of Technology, The Netherlands
\and Swansea University, United Kingdom
\\
\email{p.d.mosses@swansea.ac.uk}
}

\maketitle

\begin{abstract}

The body of this document consists of fragments of the published article.
It is for use in testing the formatting of mathematical formulae
in paragraphs, inline displays, and floating figures by \LaTeX\ and MathJax.

\end{abstract}

%%%%%%%%%%%%%%%%%%%%%%%%%%%%%%%

\section{Introduction}
\label{sec:introduction}

Version control is superfluous for funcons;
translations of language constructs to funcons, in contrast, may need to change
when the specified language evolves.
For example, the illustrative language \textsc{Imp} includes a plain old while-loop
with a Boolean-valued condition:
`$\texttt{while} \texttt{(} \VAR{BExp} \texttt{)} \VAR{Block}$'.
The following rule translates it to the funcon $\NAME{while-true}$,
which has exactly the required behaviour:
%
\begin{align*}
  \KEY{Rule} \quad
    & \SEM{execute} \LEFTPHRASE \
        \LEX{while} \ \LEX{(} \ \VAR{BExp} \ \LEX{)} \ \VAR{Block} \ \RIGHTPHRASE  = \\&\quad
      \NAME{while-true}
        (  \SEM{eval-bool} \LEFTPHRASE \ \VAR{BExp} \ \RIGHTPHRASE , 
               \SEM{execute} \LEFTPHRASE \ \VAR{Block} \ \RIGHTPHRASE  )
\end{align*}
%
The behaviour of the funcon $\NAME{while-true}$ is fixed.
But suppose the \textsc{Imp} language evolves,
and a $\VAR{Block}$ can now execute a statement `$\texttt{break;}$',
which is supposed to terminate just the \textit{closest} enclosing while-loop.
We can extend the translation with the following rule:
%
\begin{align*}
  \KEY{Rule} \quad
    & \SEM{execute} \LEFTPHRASE \ \LEX{break} \ \LEX{;} \ \RIGHTPHRASE  = 
      \NAME{abrupt}(\NAME{broken})
\end{align*}
%
The translation of `$\texttt{while(true)\{break;\}}$' is 
$\NAME{while-true}(\NAME{true}, \NAME{abrupt}(\NAME{broken}))$.
The funcon $\NAME{abrupt}(V)$ terminates execution abruptly,
signalling its argument value~$V$ as the reason for termination.
However, the behaviour of $\NAME{while-true}(\NAME{true}, X)$ is to terminate abruptly whenever $X$ does
-- so this translation would lead to abrupt termination of \emph{all} enclosing while-loops!

We cannot change the definition of $\NAME{while-true}$,
so we are forced to change the translation rule.
The following updated translation rule reflects the extension of the behaviour of while-loops
with the intended handling of abrupt termination due to break-statements,
and that they propagate abrupt termination for any other reason:
%
\begin{align*}
  \KEY{Rule} \quad
    & \SEM{execute} \LEFTPHRASE \
        \LEX{while} \ \LEX{(} \ \VAR{BExp} \ \LEX{)} \ \VAR{Block} \ \RIGHTPHRASE  = \\&\quad
      \NAME{handle-abrupt}
        (  \\&\quad\quad
        \NAME{while-true}
            (  \SEM{eval-bool} \LEFTPHRASE \ \VAR{BExp} \ \RIGHTPHRASE , 
               \SEM{execute} \LEFTPHRASE \ \VAR{Block} \ \RIGHTPHRASE  ),\\&\quad\quad
            \NAME{if-true-else}
            (   \NAME{is-equal} ( \NAME{given}, \NAME{broken}), 
               \NAME{null-value},
               \NAME{abrupt}(\NAME{given})))
\end{align*}
%
Computing $\NAME{null-value}$ represents normal termination;
$\NAME{given}$ refers to the reason for the abrupt termination.

The specialised funcon $\NAME{handle-break}$ can be used
to specify the same behaviour more concisely:
%
\begin{align*}
  \KEY{Rule} \quad
    & \SEM{execute} \LEFTPHRASE \
        \LEX{while} \ \LEX{(} \ \VAR{BExp} \ \LEX{)} \ \VAR{Block} \ \RIGHTPHRASE  = \\&\quad
      \NAME{handle-break}
        (  %\\&\quad\quad
        \NAME{while-true}
            (  \SEM{eval-bool} \LEFTPHRASE \ \VAR{BExp} \ \RIGHTPHRASE , 
               \SEM{execute} \LEFTPHRASE \ \VAR{Block} \ \RIGHTPHRASE  ))
\end{align*}
%
Wrapping $ \SEM{execute} \LEFTPHRASE \ \VAR{Block} \ \RIGHTPHRASE$ in $\NAME{handle-continue}$ 
would also support abrupt termination of the current \emph{iteration} due to executing a continue-statement.

%%%%%%%%%%%%%%%%%%%%%%%%%%%%%%%%%%%%%%%%%%%%%%%%%%%

\section{The Nature of Funcons}
\label{sec:funcons}

Funcons are often independent, but not always.
For instance, the definition of the funcon $\NAME{while-true}$
specifies the reduction of $\NAME{while-true}(B, X)$ to
a term involving the funcons $\NAME{if-true-else}$ and $\NAME{sequential}$:
%
\begin{align*}
  \KEY{Funcon} \quad
  & \NAME{while-true}(
                       B :  \TO \NAME{booleans}, X :  \TO \NAME{null-type}) 
    :  \TO \NAME{null-type} \\&\quad
    \leadsto \NAME{if-true-else}
               (  B, 
                      \NAME{sequential}
                       (  X, 
                              \NAME{while-true}
                               (  B, 
                                      X ) ), 
                      \NAME{null-value} )
\end{align*}
%

%%%%%%%%%%%%%%%%%%%%%%%%%%%%%%%%%%%%%%%%%%%%%%%%%%%

\section{Collections of Funcons}
\label{sec:collections}


%%%%%%%%%%%%%%%%%%%%%%%%%%%%%%%%%%%%%%%%%%%%%%%%%%%

\section{Facets of Funcons}
\label{sec:facets}

%%%%%%%%%%%%%%%%%%%%%%%%%%%%%%%%%%%%%%%%%%%%%%%%%%%

\section{Translation of Language Constructs to Funcons}
\label{sec:languages}

The translation specification in Fig.~\ref{fig:expressions} declares $\SYN{exp}$ as a phrase sort,
with the meta-variable $\VAR{Exp}$ (possibly with subscripts and/or primes)
ranging over phrases of that sort.
%
The BNF-like production shows two language constructs of sort $\SYN{exp}$:
an identifier of sort $\SYN{id}$
(lexical tokens, here assumed to be specified elsewhere with meta-variable $\VAR{Id}$)
and a function application written `$\VAR{Exp}_1 \texttt{(} \VAR{Exp}_2\texttt{)}$'.

\begin{figure}
    \centering
\begin{align*}
  \KEY{Syntax} \quad
    \VAR{Exp} : \SYN{exp}
      ::=  \cdots 
      \mid \SYN{id}
      \mid \SYN{exp} \ \LEX{(} \ \SYN{exp} \ \LEX{)}
      \mid \cdots
\end{align*}
\begin{align*}
  \KEY{Semantics} \quad
  & \SEM{rval} \LEFTPHRASE \ \_ : \SYN{exp} \ \RIGHTPHRASE  
    :  \TO \NAME{values} 
\\
  \KEY{Rule} \quad
    & \SEM{rval} \LEFTPHRASE \ \VAR{Id} \  \RIGHTPHRASE  = 
      \NAME{assigned-value}
        (  \NAME{bound-value}
        ( \SEM{id} \LEFTPHRASE \ \VAR{Id} \ \RIGHTPHRASE  ) )
\\
  \KEY{Rule} \quad
    & \SEM{rval} \LEFTPHRASE \ \VAR{Exp}_1 \ \LEX{(} \ \VAR{Exp}_2 \ \LEX{)} \ \RIGHTPHRASE  = 
      \NAME{apply}
        (  \SEM{rval} \LEFTPHRASE \ \VAR{Exp}_1 \ \RIGHTPHRASE , 
               \SEM{rval} \LEFTPHRASE \ \VAR{Exp}_2 \ \RIGHTPHRASE  )
\end{align*}
%
    \caption{Translation of identifiers and function applications in \textsc{Simple} to funcons}
    \label{fig:expressions}
\end{figure}

The translation specification for function declarations in Fig.~\ref{fig:declarations}
assumes a translation function $\SEM{exec}\LEFTPHRASE \VAR{Block} \RIGHTPHRASE$ 
for phrases $\VAR{Block}$ of sort $\SYN{block}$.
A block is a statement, which normally computes a null value;
but here, as in many languages, a block can return an expression value by executing a return statement,
which terminates the execution of the block abruptly.

\begin{figure}
    \centering

\begin{align*}
  \KEY{Syntax} \quad
    \VAR{Decl} : \SYN{decl}
      ::= \cdots \mid 
      \LEX{function} \ \SYN{id} \ \LEX{(} \ \SYN{id} \ \LEX{)} \ \SYN{block}
\end{align*}
\begin{align*}
  \KEY{Semantics} \quad
  & \SEM{declare} \LEFTPHRASE \ \_ : \SYN{decl} \ \RIGHTPHRASE  
    :  \TO \NAME{environments} 
\\[1ex]
  \KEY{Rule} \quad
    & \SEM{declare} \LEFTPHRASE \
        \LEX{function} \ \VAR{Id}_1 \ \LEX{(} \ \VAR{Id}_2 \ \LEX{)} \ \VAR{Block} \
                          \RIGHTPHRASE  = \\&\quad
      \NAME{bind-value}
        ( \SEM{id} \LEFTPHRASE \ \VAR{Id}_1 \ \RIGHTPHRASE , \\&\quad\quad
               \NAME{allocate-initialised-variable}
                (  \NAME{functions}
                        (  \NAME{values}, 
                           \NAME{values} ),
                    \\&\quad\quad\quad
               \NAME{function} 
                ( \NAME{closure}
                  ( \\&\quad\quad\quad\quad \NAME{scope}
                          ( \\&\quad\quad\quad\quad\quad \NAME{bind-value}
                                  (  \SEM{id} \LEFTPHRASE \ \VAR{Id}_2 \ \RIGHTPHRASE ,\\&\quad\quad\quad\quad\quad\quad
                            \NAME{allocate-initialised-variable}
                (  \NAME{values}, 
                                  \NAME{given} ) ), \\&\quad\quad\quad\quad\quad
                                 \NAME{handle-return}
                                  (  \SEM{exec} \LEFTPHRASE \ \VAR{Block} \ \RIGHTPHRASE  ) ) ) ) )
                    )
\end{align*}

    \caption{Translation of function declarations in \textsc{Simple} to funcons}
    \label{fig:declarations}
\end{figure}

%%%%%%%%%%%%%%%%%%%%%%%%%%%%%%%%%%%%%%%%%%%%%%%%%%%

\section{Defining and Implementing Funcons}
\label{sec:msos}

The funcon signature in Fig.~\ref{fig:scope} specifies that $\NAME{scope}$ takes two arguments.
The first argument is required to be pre-evaluated to a value of type $\NAME{environments}$;
the second argument should be unevaluated, as indicated by `$\TO T$'.
Values computed by $\NAME{scope}( \rho_1, X )$ are to have the same type ($T$)
as the values computed by $X$.
%
\begin{figure}
    \centering

\begin{align*}
  \KEY{Funcon} \quad
  & \NAME{scope}( \_ : \NAME{environments}, \_ :  \TO T) 
    :  \TO T 
\\
  \KEY{Rule} \quad
    & \RULE{
      \NAME{environment} (  \NAME{map-override}
                                     (  \rho_1, 
                                            \rho_0 ) ) \vdash X \TRANS 
          X'
      }{
      \NAME{environment} (  \rho_0 ) \vdash \NAME{scope}
                      (  \rho_1 : \NAME{environments}, 
                             X ) \TRANS 
          \NAME{scope}
            (  \rho_1, 
                   X' )
      }
\\
  \KEY{Rule} \quad
    & \NAME{scope}
        (  \_ : \NAME{environments}, V : T ) \leadsto V
\end{align*}

    \caption{Definition of the funcon for expressing scopes of local declrations}
    \label{fig:scope}
\end{figure}

The rules define how evaluation of $\NAME{scope}( \rho_1, X )$ can proceed
when the current bindings are represented by $\rho_0$.
%
%
The premise of the first rule holds if $X$ can make a transition to $X'$
when~$\rho_1$ overrides the current bindings~$\rho_0$.
Whether $X'$ is a computed value or an intermediate term is irrelevant.
When the premise holds, the conclusion is that $\NAME{scope}( \rho_1, X )$
can make a transition to $\NAME{scope}( \rho_1, X' )$.

%%%%%%%%%%%%%%%%%%%%%%%%%%%%%%%%%%%%%%%%%%%%%%%%%%%

\section{Related Work}
\label{sec:related}

%%%%%%%%%%%%%%%%%%%%%%%%%%%%%%%%%%%%%%%%%%%%%%%%%%%

\section{Conclusion} 
\label{sec:conclusion}

%%%%%%%%%%%%%%%%%%%%%%%%%%%%%%%%%%%%%%%%%%%%%%%%%%%

\end{document}
