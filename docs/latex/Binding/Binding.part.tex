% 



    OUTLINE
  \tableofcontents
\begin{center}
\rule{3in}{0.4pt}
\end{center}

\subsubsection{Binding}\hypertarget{binding}{}\label{binding}

\begin{align*}
  [ \
  \KEY{Type} \quad & \FUNREF{environments} \\
  \KEY{Alias} \quad & \FUNREF{envs} \\
  \KEY{Datatype} \quad & \FUNREF{identifiers} \\
  \KEY{Alias} \quad & \FUNREF{ids} \\
  \KEY{Funcon} \quad & \FUNREF{identifier-tagged} \\
  \KEY{Alias} \quad & \FUNREF{id-tagged} \\
  \KEY{Funcon} \quad & \FUNREF{fresh-identifier} \\
  \KEY{Entity} \quad & \FUNREF{environment} \\
  \KEY{Alias} \quad & \FUNREF{env} \\
  \KEY{Funcon} \quad & \FUNREF{initialise-binding} \\
  \KEY{Funcon} \quad & \FUNREF{bind-value} \\
  \KEY{Alias} \quad & \FUNREF{bind} \\
  \KEY{Funcon} \quad & \FUNREF{unbind} \\
  \KEY{Funcon} \quad & \FUNREF{bound-directly} \\
  \KEY{Funcon} \quad & \FUNREF{bound-value} \\
  \KEY{Alias} \quad & \FUNREF{bound} \\
  \KEY{Funcon} \quad & \FUNREF{closed} \\
  \KEY{Funcon} \quad & \FUNREF{scope} \\
  \KEY{Funcon} \quad & \FUNREF{accumulate} \\
  \KEY{Funcon} \quad & \FUNREF{collateral} \\
  \KEY{Funcon} \quad & \FUNREF{bind-recursively} \\
  \KEY{Funcon} \quad & \FUNREF{recursive}
  \ ]
\end{align*}
\begin{align*}
  \KEY{Meta-variables} \quad
  & \VAR{T} <: \FUNHYP{../../../Values}{Value-Types}{values}
\end{align*}
\paragraph{Environments}\hypertarget{environments}{}\label{environments}

\begin{align*}
  \KEY{Type} \quad 
  & \FUNDEC{environments}  
    \leadsto \FUNHYP{../../../Values/Composite}{Maps}{maps}
               (  \FUNREF{identifiers}, 
                      \FUNHYP{../../../Values}{Value-Types}{values}\QUERY )
\\
  \KEY{Alias} \quad
  & \FUNDEC{envs} = \FUNREF{environments}
\end{align*}
An environment represents bindings of identifiers to values.
  Mapping an identifier to $(   \  )$ represents that its binding is hidden.

Circularity in environments (due to recursive bindings) is represented using
  bindings to cut-points called $\FUNHYP{../.}{Linking}{links}$. Funcons are provided for making
  declarations recursive and for referring to bound values without explicit
  mention of links, so their existence can generally be ignored.

\begin{align*}
  \KEY{Datatype} \quad 
  \FUNDEC{identifiers} 
  \ ::= \ &
  \{ \_ : \FUNHYP{../../../Values/Composite}{Strings}{strings} \} \mid \FUNDEC{identifier-tagged}(
                     \_ : \FUNREF{identifiers}, \_ : \FUNHYP{../../../Values}{Value-Types}{values})
\end{align*}
\begin{align*}
  \KEY{Alias} \quad
  & \FUNDEC{ids} = \FUNREF{identifiers}
\\
  \KEY{Alias} \quad
  & \FUNDEC{id-tagged} = \FUNREF{identifier-tagged}
\end{align*}
An identifier is either a string of characters, or an identifier tagged with
  some value (e.g., with the identifier of a namespace).

\begin{align*}
  \KEY{Funcon} \quad
  & \FUNDEC{fresh-identifier} 
    :  \TO \FUNREF{identifiers} 
\end{align*}
$\FUNREF{fresh-identifier}$ computes an identifier distinct from all previously
  computed identifiers.

\begin{align*}
  \KEY{Rule} \quad
    & \FUNREF{fresh-identifier} \leadsto 
        \FUNREF{identifier-tagged}
          (  \STRING{generated}, 
                 \FUNHYP{../.}{Generating}{fresh-atom} )
\end{align*}
\paragraph{Current bindings}\hypertarget{current-bindings}{}\label{current-bindings}

\begin{align*}
  \KEY{Entity} \quad
  & \FUNDEC{environment}(\_ : \FUNREF{environments}) \vdash \_ \TRANS  \_
\end{align*}
\begin{align*}
  \KEY{Alias} \quad
  & \FUNDEC{env} = \FUNREF{environment}
\end{align*}
The environment entity allows a computation to refer to the current bindings
  of identifiers to values.

\begin{align*}
  \KEY{Funcon} \quad
  & \FUNDEC{initialise-binding}(
                       \VAR{X} :  \TO \VAR{T}) 
    :  \TO \VAR{T} \\&\quad
    \leadsto \FUNHYP{../.}{Linking}{initialise-linking}
               (  \FUNHYP{../.}{Generating}{initialise-generating}
                       (  \FUNREF{closed}
                               (  \VAR{X} ) ) )
\end{align*}
$\FUNREF{initialise-binding}
    (  \VAR{X} )$ ensures that $\VAR{X}$ does not depend on non-local bindings.
  It also ensures that the linking entity (used to represent potentially cyclic
  bindings) and the generating entity (for creating fresh identifiers) are 
  initialised.

\begin{align*}
  \KEY{Funcon} \quad
  & \FUNDEC{bind-value}(
                       \VAR{I} : \FUNREF{identifiers}, \VAR{V} : \FUNHYP{../../../Values}{Value-Types}{values}) 
    :  \TO \FUNREF{environments} \\&\quad
    \leadsto \{ \VAR{I} \mapsto 
                  \VAR{V} \}
\\
  \KEY{Alias} \quad
  & \FUNDEC{bind} = \FUNREF{bind-value}
\end{align*}
$\FUNREF{bind-value}
    (  \VAR{I}, 
           \VAR{X} )$ computes the environment that binds only $\VAR{I}$ to the value
  computed by $\VAR{X}$.

\begin{align*}
  \KEY{Funcon} \quad
  & \FUNDEC{unbind}(
                       \VAR{I} : \FUNREF{identifiers}) 
    :  \TO \FUNREF{environments} \\&\quad
    \leadsto \{ \VAR{I} \mapsto 
                  (   \  ) \}
\end{align*}
$\FUNREF{unbind}
    (  \VAR{I} )$ computes the environment that hides the binding of $\VAR{I}$.

\begin{align*}
  \KEY{Funcon} \quad
  & \FUNDEC{bound-directly}(
                       \_ : \FUNREF{identifiers}) 
    :  \TO \FUNHYP{../../../Values}{Value-Types}{values} 
\end{align*}
$\FUNREF{bound-directly}
    (  \VAR{I} )$ returns the value to which $\VAR{I}$ is currently bound, if any,
  and otherwise fails.

$\FUNREF{bound-directly}
    (  \VAR{I} )$ does \emph{not} follow links. It is used only in connection with
  recursively-bound values when references are not encapsulated in abstractions.

\begin{align*}
  \KEY{Rule} \quad
    & \frac{
      \FUNHYP{../../../Values/Composite}{Maps}{lookup}
        (  \VAR{\ensuremath{\rho}}, 
               \VAR{I} ) \leadsto 
        (  \VAR{V} : \FUNHYP{../../../Values}{Value-Types}{values} )
      }{
      \FUNREF{environment} (  \VAR{\ensuremath{\rho}} ) \vdash \FUNREF{bound-directly}
                    (  \VAR{I} : \FUNREF{identifiers} ) \TRANS 
        \VAR{V}
      }
\\
  \KEY{Rule} \quad
    & \frac{
      \FUNHYP{../../../Values/Composite}{Maps}{lookup}
        (  \VAR{\ensuremath{\rho}}, 
               \VAR{I} ) \leadsto 
        (   \  )
      }{
      \FUNREF{environment} (  \VAR{\ensuremath{\rho}} ) \vdash \FUNREF{bound-directly}
                    (  \VAR{I} : \FUNREF{identifiers} ) \TRANS 
        \FUNHYP{../../Abnormal}{Failing}{fail}
      }
\end{align*}
\begin{align*}
  \KEY{Funcon} \quad
  & \FUNDEC{bound-value}(
                       \VAR{I} : \FUNREF{identifiers}) 
    :  \TO \FUNHYP{../../../Values}{Value-Types}{values} \\&\quad
    \leadsto \FUNHYP{../.}{Linking}{follow-if-link}
               (  \FUNREF{bound-directly}
                       (  \VAR{I} ) )
\\
  \KEY{Alias} \quad
  & \FUNDEC{bound} = \FUNREF{bound-value}
\end{align*}
$\FUNREF{bound-value}
    (  \VAR{I} )$ inspects the value to which $\VAR{I}$ is currently bound, if any,
   and otherwise fails. If the value is a link, $\FUNREF{bound-value}
    (  \VAR{I} )$ returns the
   value obtained by following the link, if any, and otherwise fails. If the 
   inspected value is not a link, $\FUNREF{bound-value}
    (  \VAR{I} )$ returns it.

$\FUNREF{bound-value}
    (  \VAR{I} )$ is used for references to non-recursive bindings and to
   recursively-bound values when references are encapsulated in abstractions.

\paragraph{Scope}\hypertarget{scope}{}\label{scope}

\begin{align*}
  \KEY{Funcon} \quad
  & \FUNDEC{closed}(
                       \VAR{X} :  \TO \VAR{T}) 
    :  \TO \VAR{T} 
\end{align*}
$\FUNREF{closed}
    (  \VAR{X} )$ ensures that $\VAR{X}$ does not depend on non-local bindings.

\begin{align*}
  \KEY{Rule} \quad
    & \frac{
      \FUNREF{environment} (  \FUNHYP{../../../Values/Composite}{Maps}{map}
                                   (   \  ) ) \vdash \VAR{X} \TRANS 
        \VAR{X}'
      }{
      \FUNREF{environment} (  \_ ) \vdash \FUNREF{closed}
                    (  \VAR{X} ) \TRANS 
        \FUNREF{closed}
          (  \VAR{X}' )
      }
\\
  \KEY{Rule} \quad
    & \FUNREF{closed}
        (  \VAR{V} : \VAR{T} ) \leadsto 
        \VAR{V}
\end{align*}
\begin{align*}
  \KEY{Funcon} \quad
  & \FUNDEC{scope}(
                       \_ : \FUNREF{environments}, \_ :  \TO \VAR{T}) 
    :  \TO \VAR{T} 
\end{align*}
$\FUNREF{scope}
    (  \VAR{D}, 
           \VAR{X} )$ executes $\VAR{D}$ with the current bindings, to compute an environment
  $\VAR{\ensuremath{\rho}}$ representing local bindings. It then executes $\VAR{X}$ to compute the result,
  with the current bindings extended by $\VAR{\ensuremath{\rho}}$, which may shadow or hide previous
  bindings.

$\FUNREF{closed}
    (  \FUNREF{scope}
            (  \VAR{\ensuremath{\rho}}, 
                   \VAR{X} ) )$ ensures that $\VAR{X}$ can reference only the bindings
  provided by $\VAR{\ensuremath{\rho}}$.

\begin{align*}
  \KEY{Rule} \quad
    & \frac{
      \FUNREF{environment} (  \FUNHYP{../../../Values/Composite}{Maps}{map-override}
                                   (  \VAR{\ensuremath{\rho}}\SUB{1}, 
                                          \VAR{\ensuremath{\rho}}\SUB{0} ) ) \vdash \VAR{X} \TRANS 
        \VAR{X}'
      }{
      \FUNREF{environment} (  \VAR{\ensuremath{\rho}}\SUB{0} ) \vdash \FUNREF{scope}
                    (  \VAR{\ensuremath{\rho}}\SUB{1} : \FUNREF{environments}, 
                           \VAR{X} ) \TRANS 
        \FUNREF{scope}
          (  \VAR{\ensuremath{\rho}}\SUB{1}, 
                 \VAR{X}' )
      }
\\
  \KEY{Rule} \quad
    & \FUNREF{scope}
        (  \_ : \FUNREF{environments}, 
               \VAR{V} : \VAR{T} ) \leadsto 
        \VAR{V}
\end{align*}
\begin{align*}
  \KEY{Funcon} \quad
  & \FUNDEC{accumulate}(
                       \_ : (   \TO \FUNREF{environments} )\STAR) 
    :  \TO \FUNREF{environments} 
\end{align*}
$\FUNREF{accumulate}
    (  \VAR{D}\SUB{1}, 
           \VAR{D}\SUB{2} )$ executes $\VAR{D}\SUB{1}$ with the current bindings, to compute an
  environment $\VAR{\ensuremath{\rho}}\SUB{1}$ representing some local bindings. It then executes $\VAR{D}\SUB{2}$ to
  compute an environment $\VAR{\ensuremath{\rho}}\SUB{2}$ representing further local bindings, with the
  current bindings extended by $\VAR{\ensuremath{\rho}}\SUB{1}$, which may shadow or hide previous
  current bindings. The result is $\VAR{\ensuremath{\rho}}\SUB{1}$ extended by $\VAR{\ensuremath{\rho}}\SUB{2}$, which may shadow
  or hide the bindings of $\VAR{\ensuremath{\rho}}\SUB{1}$.

$\FUNREF{accumulate}
    (  \_, 
           \_ )$ is associative, with $\FUNHYP{../../../Values/Composite}{Maps}{map}
    (   \  )$ as unit, and extends to any
  number of arguments.

\begin{align*}
  \KEY{Rule} \quad
    & \frac{
       \VAR{D}\SUB{1} \TRANS 
        \VAR{D}\SUB{1}'
      }{
       \FUNREF{accumulate}
                    (  \VAR{D}\SUB{1}, 
                           \VAR{D}\SUB{2} ) \TRANS 
        \FUNREF{accumulate}
          (  \VAR{D}\SUB{1}', 
                 \VAR{D}\SUB{2} )
      }
\\
  \KEY{Rule} \quad
    & \FUNREF{accumulate}
        (  \VAR{\ensuremath{\rho}}\SUB{1} : \FUNREF{environments}, 
               \VAR{D}\SUB{2} ) \leadsto 
        \FUNREF{scope}
          (  \VAR{\ensuremath{\rho}}\SUB{1}, 
                 \FUNHYP{../../../Values/Composite}{Maps}{map-override}
                  (  \VAR{D}\SUB{2}, 
                         \VAR{\ensuremath{\rho}}\SUB{1} ) )
\\
  \KEY{Rule} \quad
    & \FUNREF{accumulate}
        (   \  ) \leadsto 
        \FUNHYP{../../../Values/Composite}{Maps}{map}
          (   \  )
\\
  \KEY{Rule} \quad
    & \FUNREF{accumulate}
        (  \VAR{D}\SUB{1} ) \leadsto 
        \VAR{D}\SUB{1}
\\
  \KEY{Rule} \quad
    & \FUNREF{accumulate}
        (  \VAR{D}\SUB{1}, 
               \VAR{D}\SUB{2}, 
               \VAR{D}\PLUS ) \leadsto 
        \FUNREF{accumulate}
          (  \VAR{D}\SUB{1}, 
                 \FUNREF{accumulate}
                  (  \VAR{D}\SUB{2}, 
                         \VAR{D}\PLUS ) )
\end{align*}
\begin{align*}
  \KEY{Funcon} \quad
  & \FUNDEC{collateral}(
                       \VAR{\ensuremath{\rho}}\STAR : \FUNREF{environments}\STAR) 
    :  \TO \FUNREF{environments} \\&\quad
    \leadsto \FUNHYP{../../Abnormal}{Failing}{checked} \ 
               \FUNHYP{../../../Values/Composite}{Maps}{map-unite}
                 (  \VAR{\ensuremath{\rho}}\STAR )
\end{align*}
$\FUNREF{collateral}
    (  \VAR{D}\SUB{1}, 
           \cdots )$ pre-evaluates its arguments with the current bindings,
  and unites the resulting maps, which fails if the domains are not pairwise
  disjoint.

$\FUNREF{collateral}
    (  \VAR{D}\SUB{1}, 
           \VAR{D}\SUB{2} )$ is associative and commutative with $\FUNHYP{../../../Values/Composite}{Maps}{map}
    (   \  )$ as unit, 
  and extends to any number of arguments.

\paragraph{Recurse}\hypertarget{recurse}{}\label{recurse}

\begin{align*}
  \KEY{Funcon} \quad
  & \FUNDEC{bind-recursively}(
                       \VAR{I} : \FUNREF{identifiers}, \VAR{E} :  \TO \FUNHYP{../../../Values}{Value-Types}{values}) 
    :  \TO \FUNREF{environments} \\&\quad
    \leadsto \FUNREF{recursive}
               (  \{  \VAR{I} \}, 
                      \FUNREF{bind-value}
                       (  \VAR{I}, 
                              \VAR{E} ) )
\end{align*}
$\FUNREF{bind-recursively}
    (  \VAR{I}, 
           \VAR{E} )$ binds $\VAR{I}$ to a link that refers to the value of $\VAR{E}$, 
  representing a recursive binding of $\VAR{I}$ to the value of $\VAR{E}$.
  Since $\FUNREF{bound-value}
    (  \VAR{I} )$ follows links, it should not be executed during the
  evaluation of $\VAR{E}$.

\begin{align*}
  \KEY{Funcon} \quad
  & \FUNDEC{recursive}(
                       \VAR{SI} : \FUNHYP{../../../Values/Composite}{Sets}{sets}
                                 (  \FUNREF{identifiers} ), \VAR{D} :  \TO \FUNREF{environments}) 
    :  \TO \FUNREF{environments} \\&\quad
    \leadsto \FUNREF{re-close}
               (  \FUNREF{bind-to-forward-links}
                       (  \VAR{SI} ), 
                      \VAR{D} )
\end{align*}
$\FUNREF{recursive}
    (  \VAR{SI}, 
           \VAR{D} )$ executes $\VAR{D}$ with potential recursion on the bindings of 
  the identifiers in the set $\VAR{SI}$ (which need not be the same as the set of
  identifiers bound by $\VAR{D}$).

\begin{align*}
  \KEY{Auxiliary Funcon} \quad
  & \FUNDEC{re-close}(
                       \VAR{M} : \FUNHYP{../../../Values/Composite}{Maps}{maps}
                                 (  \FUNREF{identifiers}, 
                                        \FUNHYP{../.}{Linking}{links} ), \VAR{D} :  \TO \FUNREF{environments}) 
    :  \TO \FUNREF{environments} \\&\quad
    \leadsto \FUNREF{accumulate}
               (  \FUNREF{scope}
                       (  \VAR{M}, 
                              \VAR{D} ), 
                      \FUNHYP{../.}{Flowing}{sequential}
                       (  \FUNREF{set-forward-links}
                               (  \VAR{M} ), 
                              \FUNHYP{../../../Values/Composite}{Maps}{map}
                               (   \  ) ) )
\end{align*}
$\FUNREF{re-close}
    (  \VAR{M}, 
           \VAR{D} )$ first executes $\VAR{D}$ in the scope $\VAR{M}$, which maps identifiers
  to freshly allocated links. This computes an environment $\VAR{\ensuremath{\rho}}$ where the bound
  values may contain links, or implicit references to links in abstraction
  values. It then sets the link for each identifier in the domain of $\VAR{M}$ to
  refer to its bound value in $\VAR{\ensuremath{\rho}}$, and returns $\VAR{\ensuremath{\rho}}$ as the result.

\begin{align*}
  \KEY{Auxiliary Funcon} \quad
  & \FUNDEC{bind-to-forward-links}(
                       \VAR{SI} : \FUNHYP{../../../Values/Composite}{Sets}{sets}
                                 (  \FUNREF{identifiers} )) 
    :  \TO \FUNHYP{../../../Values/Composite}{Maps}{maps}
                     (  \FUNREF{identifiers}, 
                            \FUNHYP{../.}{Linking}{links} ) \\&\quad
    \leadsto \FUNHYP{../../../Values/Composite}{Maps}{map-unite}
               ( \\&\quad\quad\quad\quad \FUNHYP{../.}{Giving}{interleave-map}
                       ( \\&\quad\quad\quad\quad\quad \FUNREF{bind-value}
                               (  \FUNHYP{../.}{Giving}{given}, 
                                      \FUNHYP{../.}{Linking}{fresh-link}
                                       (  \FUNHYP{../../../Values}{Value-Types}{values} ) ), \\&\quad\quad\quad\quad\quad
                              \FUNHYP{../../../Values/Composite}{Sets}{set-elements}
                               (  \VAR{SI} ) ) )
\end{align*}
$\FUNREF{bind-to-forward-links}
    (  \VAR{SI} )$ binds each identifier in the set $\VAR{SI}$ to a
  freshly allocated link.

\begin{align*}
  \KEY{Auxiliary Funcon} \quad
  & \FUNDEC{set-forward-links}(
                       \VAR{M} : \FUNHYP{../../../Values/Composite}{Maps}{maps}
                                 (  \FUNREF{identifiers}, 
                                        \FUNHYP{../.}{Linking}{links} )) 
    :  \TO \FUNHYP{../../../Values/Primitive}{Null}{null-type} \\&\quad
    \leadsto \FUNHYP{../.}{Flowing}{effect}
               ( \\&\quad\quad\quad\quad \FUNHYP{../.}{Giving}{interleave-map}
                       ( \\&\quad\quad\quad\quad\quad \FUNHYP{../.}{Linking}{set-link}
                               (  \FUNHYP{../../../Values/Composite}{Maps}{map-lookup}
                                       (  \VAR{M}, 
                                              \FUNHYP{../.}{Giving}{given} ), 
                                      \FUNREF{bound-value}
                                       (  \FUNHYP{../.}{Giving}{given} ) ), \\&\quad\quad\quad\quad\quad
                              \FUNHYP{../../../Values/Composite}{Sets}{set-elements}
                               (  \FUNHYP{../../../Values/Composite}{Maps}{map-domain}
                                       (  \VAR{M} ) ) ) )
\end{align*}
For each identifier $\VAR{I}$ in the domain of $\VAR{M}$, $\FUNREF{set-forward-links}
    (  \VAR{M} )$ sets the 
  link to which $\VAR{I}$ is mapped by $\VAR{M}$ to the current bound value of $\VAR{I}$.

% 


